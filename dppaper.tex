% !TeX root = RJwrapper.tex
\title{dapper: Data Augmentation for Private Posterior Estimation in R}


\author{by Kevin Eng, Jordan A. Awan, Ruobin Gong, Nianqiao Phyllis Ju, and Vinayak A. Rao}

\maketitle

\abstract{%
This paper serves as a reference and introduction on using the dapper R package. The goal of this package is to provide some tools for exploring the impact of different privacy regimes on a Bayesian analysis. A strength of this framework is the ability to target the exact posterior in settings where the likelihood is too complex to analytically express.
}

\hypertarget{introduction}{%
\section{Introduction}\label{introduction}}

Differential privacy provides a rigorous framework for protecting
confidential information. In this framework, privacy is obtained
through the injection of random noise in the data analysis workflow.
It has served as the theoretical foundation for recent advances in privacy technology. Several high profile
examples include Apple (), Google (), Microsoft (), and the
U.S. Census Bureau ().

There are several approaches for constructing or modifying a data analysis workflow
to provide privacy guarantees, which loosely fall into three categories: direct, query, and dissemination.
The approaches mainly differ in where in the workflow privacy noise is injected.
In the dissemination setting, data needs to be released to the public, and privacy is maintained by first
perturbing the data with random noise before being released. There is strong interest
in further developing this approach because data curators are often interested
in releasing data to the public without knowledge of how their data will
be used or analyzed. In this scenario, it can be impossible to apply the direct or
query method because they require knowing before hand how the data will be analyzed.
For a recent example of a high profile application of the dissemination approach
see the the U.S. Census Bureau's \texttt{TopDown} algorithm (Bureau 2023). (Ji, Lipton, and Elkan 2014)
provide a nice survey of the direct and query based methods applied to
common machine learning algorithms.

Correctly carrying out statistical inference in the dissemination setting
requires adjusting statistical workflows to account for privacy noise.
As an example, one instance of implementing the dissemination approach for tabular data involves directly
adding independent, random error to each cell. For regression models,
this corresponds to having data with measurement errors
in the covariates. This, unfortunately, violates the assumptions of most statistical models.
In the presence of such errors, standard estimators can exhibit significant bias (Gong 2022).
Therefore, fitting models without accounting for the added privacy noise can lead to incorrect inference.

Unfortunately, making the necessary adjustments poses formidable mathematical
challenges, even for seemingly simple models like linear regression.
The difficulty can be seen from considering the marginal likelihood that results from correctly accounting for the injected
privacy noise. This function is often analytically intractable and as a result,
it is difficult or impossible to apply common maximum likelihood methods
to derive estimators. In particular, the marginal likelihood can involve a complex
integral where it is not possible to even evaluate the likelihood
at a point. This makes gradient based methods, like the Newton-Raphson method,
impractical. And approximating the likelihood can be computationally
unfeasible since the integral is also of high dimension.
Few tools are available to researchers to address these issues,
and their absence is a serious barrier to the wider adoption
of dissemination methods.

The DAPPER package provides a tool for conducting
valid statistical inference in the presence of privacy noise.
It implements the Bayesian framework proposed in (Ju et al. 2022). This framework describes how to modify
an existing Bayesian model to account for privacy noise. DAPPER
serves as user-friendly interface for implementing the framework in R. It allows the
user to specify a sampler from an existing Bayesian model and
automatically constructing a valid posterior sampler that accounts for the
added privacy noise.

In the previously mentioned case where privacy noise is
directly added to each cell of a data table, there exist
readily available methods from measurement error literature for making adjustments.
For textbook length treatments on the topic see (Yi 2017; Carroll et al. 2006).
Work in this area mostly focuses on methods which do not require fully specifying the
measurement error model, since this is often assumed unknown.
However, in differential privacy, the measurement error model is exactly known.
This difference, makes feasible some ideas which the measurement
error community has not previously considered (Smith 2011; Karwa, Kifer, and Slavković 2015).

The rest of this article is organized as follows. Section 2 covers the necessary background to understand the mathematical notation
and ideas used throughout the paper. Section 3 goes over the main algorithm without
going into mathematical detail, for specifics see (Ju et al. 2022). Section 4 provides
an overview of the dapper package and discusses important implementation details.
Section 5 contains two example of how one might use the package to analyze the
impact of adding noise for privacy. The first example goes over a typical
odds ratio analysis for a \(2 \times 2\) table and the second example
covers a linear regression model.

\hypertarget{background}{%
\section{Background}\label{background}}

Let \(x = (x_1, \ldots, x_n) \in \mathcal{X}^n\) represent a confidential
database containing \(n\) records. Usually, the goal of collecting data
is to learn some characteristic about the underlying population.
To accomplish this task, a common approach is to assume the population
is represented by some statistical model \(f( \cdot \mid \theta)\). It is often the case that
some function of \(\theta\) has relevant meaning to the scientific question at hand. In this setting,
learning characteristics of a population reduces to learning about \(\theta\).

In the Bayesian statistical framework, learning about \(\theta\) is accomplished by drawing samples from the
posterior \(p(\theta \mid x) \propto f(x \mid \theta) p(\theta)\). For
large data sets, it is common to work with a summary statistic \(s = s(x)\)
that has much smaller dimension than the original data. Doing so can
greatly simplify calculations. In general, there can be information
loss with using summary statistics, but for models with a sufficient
statistic, there is no loss. Curators of large databases
often use summary statistics to publicize data since it allows them
to efficiently communicate information contained in large data sets.
For this reason, summary statistics are a natural target for
dissemination based privacy approaches.

\hypertarget{differential-privacy}{%
\subsection{Differential Privacy}\label{differential-privacy}}

While a summary statistic can already partially anonymize data, it is still
possible to deduce information about an individual entry depending on
the distribution of \(x\). Differential privacy
solves this problem by taking a summary statistic \(s\), and adding noise to it to produce a noisy summary statistic \(s_{dp}\).
While this method is not new {[}insert citation on swapping and randomized responses{]}, differential privacy
provides a rigorous framework for specify where and how much
noise to add.

We now describe the \(\epsilon\)-DP privacy framework in more detail. For the noisy summary
statistic, we write \(s_{dp} \sim \eta(\cdot \mid x)\). Here,
\(\eta\) is a known noise infusion process designed to meet a certain property: The privacy mechanism
\(\eta\) is said to be \(\epsilon\)-differentially private (Dwork and Roth 2013) if for all values of
\(s_{dp}\), and all ``neighboring'' databases \((x,x') \in \mathcal{X}^n \times \mathcal{X}^n\) differing
by one record (denoted by \(d(x,x')\)), the probability ratio is bounded:
\[
\dfrac{\eta(s_{dp} \mid x)}{\eta(s_{dp} \mid x')} \leq \exp(\epsilon), \quad \epsilon > 0.
\]
The parameter \(\epsilon\) is called the privacy loss budget, and controls how
strong the privacy guarantee is. Larger values of \(\epsilon\) correspond to weaker
privacy guarantees which in turn means less noise being added.

The differential privacy framework is used to create and verify privacy
mechanisms. One such mechanism is the Laplace mechanism. It works by
taking a deterministic query \(s: X \mapsto \mathbb{R}^m\) and constructing
the privatized query \(s_{dp} := s(x) + u\) where \(u\) is an \(m\)-dimensional
vector of iid Laplace noise. Using the ratio bound, if we draw
each \(u_i \sim Lap(\Delta (s) / \epsilon)\), we can show \(s_dp\) is \(\epsilon\)-differentially private.
Here \(\Delta (s)\) is the global sensitivity of \(s\). Example 2, will cover an
application of the Laplace mechanism to linear regression.

\hypertarget{data-augmentation}{%
\subsection{Data Augmentation}\label{data-augmentation}}

The idea behind data augmentation is to run a Gibbs sampler on a coupling of two random variables where one
of the marginals is the target distribution. Suppose we wish to sample from a density \(f_{A}\) which is difficult.
The data augmentation method instead considers sampling from a joint distribution \(f(a,b)\). Since
we are ultimately interested in samples from \(f_A\), the joint
distribution should be chosen so that (i) the marginal distribution
with respects to \(a\) is \(f_{A}\) and (ii) \(f(a \mid b)\) and \(f(b \mid a)\)
are easy to sample from. The choice \(f\) is not unique and can require
some foresight.

\hypertarget{methodology}{%
\section{Methodology}\label{methodology}}

Given data \(s_{dp}\), the goal of Bayesian inference is to sample from the
posterior distribution \(p(\theta \mid s_{dp})\). Since the observed likelihood,
\(p(s_{dp} \mid \theta)\), often has no simple closed form expression, most standard
sampling schemes do not apply. To conduct privacy-aware Bayesian inference, the dapper package implements
the data augmentation algorithm which allows us to sample from \(p(\theta \mid s_{dp})\)
without needing to specify \(p(s_{dp} \mid \theta)\).

The algorithm considers the joint distribution \(p(\theta, x \mid s_{dp})\) and
alternates sampling from the two distributions

\begin{itemize}
\tightlist
\item
  \(p(\theta \mid x, s_{dp})\)
\item
  \(p(x \mid \theta, s_{dp})\)
\end{itemize}

Since \(s_{dp}\) is derived from \(x\), we have \(p(\theta \mid x, s_{dp}) = p(\theta \mid x)\) which
is just the usual posterior distribution given the confidential data \(x\). The dapper
package assumes the user has access to a sampler for \(p(\theta \mid x)\). This can
come from any R package such as fmcmc. For the second distribution, \(p(x \mid \theta, s_{dp})\), may
only be known up to a constant. The dapper package samples from this distribution by
running a Gibbs-like sampler. Each of the \(n\) components of \(x\) is individually
updated. However unlike the standard Gibbs sampler, each component is updated
using a Metropolis-Hasting algorithm. This method is sometimes called the Metropolis within Gibbs sampler (Robert and Casella 2004).

In some cases, sampling from \(p(x \mid \theta, s_{dp})\) can be made more efficient
when the privacy mechanism can be written as a function of \(s_{dp}\) and
a sum consisting of contribution from each individual record. More precisely, we say the privacy mechanism satisfies
the record additivity property if
\[
\eta(s_{dp} \mid x) = g(s_{dp}, \sum_{i=1}^{n}t_i(x_i, s_{dp}))
\]
for some known and tractable functions \(g, t_1, \ldots, t_n\). The sample mean is a
example of a summary statistic satisfying record additivity where \(t_i(x_i, s_{dp}) = x_i\).

The algorithm is in the following pseudo code:

\begin{enumerate}
\def\labelenumi{\arabic{enumi}.}
\tightlist
\item
  Sample \(\theta^{t+1}\) from \(p(\cdot \mid x^{(t)})\).
\item
  Sample from \(p(x \mid \theta, s_{dp})\) using a three step process

  \begin{itemize}
  \tightlist
  \item
    Propose \(x_{i}^{*} \sim f(\cdot \mid \theta)\).
  \item
    If \(s\) satisfies the record additive property then
    update \(s(x^*, s_{dp}) = t(x,s_{dp}) - t_i(x_i,s_{dp}) + t_{i}(x_i^*, s_{dp})\).
  \item
    Accept the proposed state with probability \(\alpha(x_i^* \mid x_i, x_{-i}, \theta)\)
    given by:
  \end{itemize}

  \[
     \alpha(x_i^* \mid x_i, x_{-i}, \theta) = \min \left\{ 1, \dfrac{\eta(s_{dp} \mid x_i^*, x_{-i})}{\eta(s_{dp} \mid x_i, x_{-i})} \right\}.
   \]
\end{enumerate}

\hypertarget{the-structure-of-dapper}{%
\section{The Structure of dapper}\label{the-structure-of-dapper}}

The package is structured around the two functions \texttt{dapper\_sample} and
\texttt{new\_privacy}. The first function is used to draw samples from the
posterior. The second function is used to create a privacy data model object
which the \texttt{dapper\_sample} function requires as input. The purpose of the data model
object is to collect all the components specific to the data augmentation algorithm
into one bundle. This way, the other arguments into \texttt{dapper\_sample} pertain only
to sampling parameters such as the number of iterations.

Since the input to these functions are R functions, there is a great deal of freedom
in implementation. The next two sections describe in detail the inputs into
these functions and highlight some considerations that should be taken
into account in order avoid slow or unexpected behavior.

Before delving into the specifics of each component, it is necessary to clearly
define how the confidential data is represented. Internally, the
confidential database is encoded as a 2D matrix. There can be multiple ways
of doing this and is often the case when dealing with likelihood functions
that has a sufficient statistic. For example, if our data consist
of 100 responses from a two question, yes/no, survey. Then we can either encode
the data as a \(2 \times 2\) matrix, or a \(100 \times 2\) matrix. Both are mathematically
equivalent, but the \(2 \times 2\) matrix will be much more memory efficient.
In general, the representation that uses the least amount of memory should be
used. Correctly specifying the privacy model will require a consistent
representation among all of its components.

\hypertarget{privacy-model}{%
\subsection{Privacy Model}\label{privacy-model}}

Creating a privacy model is done using the \texttt{new\_privacy} constructor. The
main arguments consist of the four components as outlined in the methodology
section.

\begin{verbatim}
new_privacy(post_f = NULL, latent_f = NULL, priv_f = NULL,
            st_f = NULL, add = FALSE, npar = NULL)
\end{verbatim}

The internal implementation of the DA algorithm in \texttt{dapper\_sample} requires
some care in how each component is constructed.

\begin{itemize}
\item
  \texttt{latent\_f} is an R function that samples from the latent process. Its
  syntax should be \texttt{lik\_smpl(theta)} where \texttt{theta} is a vector
  representing the likelihood model parameters being estimated. This function
  must work with the supplied initial parameter provide in the \texttt{init\_par}
  argument of \texttt{dapper\_sample} function. The output should be a \(n \times k\) matrix
  where \(k\) is the dimension of the vector of likelihood parameters.
\item
  \texttt{post\_f} is a function which makes draws from the posterior sampler. It should
  have the syntax \texttt{post\_f(dmat,\ theta)}. Here \texttt{dmat} is the
  hypothetical data set representing the confidential data. This sampler can be generated by wrapping mcmc samplers generated from other R packages
  (e.g.~\CRANpkg{rstan}, \CRANpkg{fmcmc}, \CRANpkg{adaptMCMC}).
  If using this approach, it is recommended to avoid using packages
  with a large initialization overhead such as \CRANpkg{mcmc} since the sampler is reinitialized
  every loop iteration. In the case of \CRANpkg{mcmc},
  the Metropolis-Hastings loop is implemented in C so there is a significant initialization cost
  when calling from an R function. The purpose of the \texttt{theta} argument is
  to serve as the initialization point if samples from \texttt{post\_f} are draws
  from say a Metropolis-Hastings sampler.
\item
  \texttt{priv\_f} is an R function that represents the log of the privacy mechanism density, \(\eta(s_{sdp} \mid x)\).
\item
  \texttt{st\_f} is an R function which calculates a summary statistic. The optional
  argument \texttt{add} is a flag which represents whether \(st_f\) should be interpreted as
  satisfying record additivity or not.
\end{itemize}

\hypertarget{sampling}{%
\subsection{Sampling}\label{sampling}}

The main function in \pkg{DAPPER} is the \texttt{dapper\_sample} function. The syntax of the function is:

\begin{verbatim}
dapper_sample(data_model, sdp, nobs, init_par, niter = 2000, warmup = floor(niter / 2),
           chains = 1, varnames = NULL)
\end{verbatim}

The three required inputs into \texttt{dapper\_sample} function are the privacy model (\texttt{data\_model}), the value
of the observed privatized statistic (\texttt{sdp}), and the total number of observations
in the complete data (\texttt{nobs}). The dapper
package is best suited for problems where the complete data can be represented in
tabular form. This is because internally, it is represented as a matrix.

The optional arguments are the number of mcmc draws (\texttt{niter}), the
burn in period (\texttt{warmup}), number of chains (\texttt{chains}) and character
vector that names the parameters. Running multiple chains can be done in parallel
using the \CRANpkg{furrr} package. Additionally, progress can be monitored
using the \CRANpkg{progressr} package.

The \texttt{data\_model} input is a \texttt{privacy}
object that can be constructed using the \texttt{new\_privacy} constructor. The
process of constructing a \texttt{privacy} object will be discussed in the next section.

\hypertarget{examples}{%
\section{Examples}\label{examples}}

\hypertarget{x2-contingency-table}{%
\subsection{2x2 Contingency Table}\label{x2-contingency-table}}

As a demonstration, we analyze the UC Berkeley admissions data, which is often
used as an illustrative example of Simpson's paradox. The question posed is whether
the data suggest there is bias against females during the college admissions
process. Below is a table of the aggregate admissions result from six departments based on sex
for a total of \(N = 4526\) applicants. The table on the left represents
the true admissions data and the table on the right is the result of adding
independent, \(N(0,100^2)\) error to each cell. Throughout the example
we assume we only have access to the original total count, \(N\), and
the noise infused table.

\begin{table}[!h]

\centering
\begin{tabular}[t]{lrr}
\toprule
  & Male & Female\\
\midrule
Admitted & 1198 & 557\\
Rejected & 1493 & 1278\\
\bottomrule
\end{tabular}
\centering
\begin{tabular}[t]{lrr}
\toprule
  & Male & Female\\
\midrule
Admitted & 1135 & 473\\
Rejected & 1511 & 1438\\
\bottomrule
\end{tabular}
\end{table}

Below we walk through the process of defining a privacy model.

\begin{enumerate}
\def\labelenumi{\arabic{enumi}.}
\item
  \texttt{latent\_f}: Since we can condition on the table total \(N\), we can model the original, unobserved table counts as a multinomial
  distribution. We can easily draw from this distribution using the
  \texttt{rmultinom} function in the \texttt{base} stats package. Note, in this example,
  the return value of one sample from \texttt{rmultinom} is a \(4 \times 1\) matrix. In order to
  conform with \texttt{dapper\_sample} we must convert the matrix to a vector.

\begin{verbatim}
latent_f <- function(theta) {
  t(rmultinom(1, 4526, theta))
}
\end{verbatim}
\item
  \texttt{post\_f}: Given confidential data, we can derive the posterior analytically
  using a Dirichlet prior. In this example, we use a flat prior which
  corresponds to Dirch(1) distribution. A sample from the Dirichlet distribution
  can be generated using random draws from the gamma distribution.

\begin{verbatim}
post_f <- function(dmat, theta) {
  x <- c(dmat)
  t1 <- rgamma(4, x + 1, 1)
  t1/sum(t1)
}
\end{verbatim}
\item
  \texttt{st\_f}: Since the latent model only returns one observation from a multinomial
  distribution, we can just use the identity function as the summary statistic.

\begin{verbatim}
st_f <- function(dmat) {
  c(dmat)
}
\end{verbatim}
\item
  \texttt{priv\_f}: The privacy mechanism is Guassian white noise drawn from independent \(N(0,100^2)\) distributions. Hence given
  confidential table cells \((n_{11}, n_{22}, n_{12}, n_{21})\)
  \[
  \eta(s_{dp} \mid x) = \prod \phi(s_{sd}; n_{ij}, 100^2).
  \]
  Here \(\phi(\cdot;\mu,\sigma^2)\) is the density of the normal distribution
  with mean and variance \(\mu,\sigma^2\).

\begin{verbatim}
priv_f <- function(sdp, x) {
  dnorm(sdp - x, mean = 0, sd = 100, log = TRUE)
}
\end{verbatim}
\end{enumerate}

Once we have defined all components of the model we can
create a new privacy model object using the \texttt{new\_privacy} function and
feed this into the \texttt{dapper\_sample} function. Below we simulate 10,000 posterior
draws with a burn-in of 1000.

\begin{verbatim}
library(dapper)
dmod <- new_privacy(post_f   = post_f,
                    latent_f = latent_f,
                    priv_f   = priv_f,
                    st_f     = st_f,
                    add      = FALSE,
                    npar     = 4,
                    varnames = c("pi_11", "pi_21", "pi_12", "pi_22"))
                  
dp_out <- dapper_sample(dmod,
                  sdp = c(adm_prv),
                  niter = 10000,
                  warmup = 1000,
                  chains = 1,
                  init_par = rep(.25,4))
\end{verbatim}

results can be quickly summarized using the \texttt{summary} function which is
displayed below. The \texttt{rhat} values in the table are close to 1, which indicates
the chain has run long enough to achieve adequate mixing.

\begin{verbatim}
#> # A tibble: 4 x 10
#>   variable  mean median     sd    mad     q5   q95  rhat ess_bulk ess_tail
#>   <chr>    <num>  <num>  <num>  <num>  <num> <num> <num>    <num>    <num>
#> 1 pi_11    0.248  0.249 0.0254 0.0257 0.206  0.288  1.00    191.      388.
#> 2 pi_21    0.334  0.333 0.0261 0.0254 0.291  0.377  1.00    210.      329.
#> 3 pi_12    0.103  0.103 0.0272 0.0271 0.0586 0.149  1.02     76.0     159.
#> 4 pi_22    0.315  0.315 0.0256 0.0261 0.275  0.358  1.00    164.      460.
\end{verbatim}

Diagnostic checks using trace plots can be done using the \pkg{Bayesplot} package
as shown in figure {[}x{]}. It is especially important to check for good mixing
with \texttt{dapper} since sticky chains are likely to be produced
when the amount of injected noise is high. See the discussion section
for a more detailed explanation.

\begin{center}\includegraphics{dppaper_files/figure-latex/unnamed-chunk-12-1} \end{center}

To see if there is evidence of gender bias we can look at the odds ratio.
Specifically, we look at the odds of a male being admitted to
that of female. A higher odds ratio would indicate a bias
favoring males. Figure {[}Z{]} shows the posterior draws
from the dapper model. The large odds ratio values would seem
to indicate there is bias favoring the males. See {[}XXX{]} for
an explanation of the ``paradox'' of this result.

\begin{center}\includegraphics{dppaper_files/figure-latex/unnamed-chunk-13-1} \end{center}

For comparison, we run a standard Bayesian analysis on the
noise infused table ignoring the privacy mechanism. This will
correspond exactly to the model defined in the \texttt{post\_f} component.
Figure {[}X{]} shows a density estimate for the odds ratio
under the confidential and noisy data. The posterior
distribution for the odds ratio under the noisy data
is shifted significantly, indicating a large degree of bias.
Looking at figure {[}Y{]} shows the MAP estimate from DAPPER
is similar to that in the case of the confidential data.
The width of the posterior is also much larger since
it properly accounts for the uncertainty due to the privacy mechanism. This
illustrates the dangers of ignoring the privacy mechanism. The naive
analysis not only has bias, but also severely underestimate the
uncertainty associated with the odds ratio estimate.

\begin{center}\includegraphics{dppaper_files/figure-latex/unnamed-chunk-14-1} \end{center}

\hypertarget{linear-regression}{%
\subsection{Linear Regression}\label{linear-regression}}

Below we review an application of \texttt{dapper} to an example
presented in (Ju et al. 2022) where they apply a Laplace privacy
mechanism to a sufficient summary statistic for a linear regression model.
Let \(\{(x_i,y_i)\}_{i=1}^{n}\) be the original, confidential data with \(x_i \in \mathbb{R}^2\).
They assume the true data generating process follows the model

\[
\begin{aligned}
y &= -1.79 -2.89x_2 -0.66x_2 + \epsilon\\
\epsilon &\sim N(0,2^2)
\end{aligned}
\]

and consider the scenario where one desires to publicly release the
sufficient summary statistics

\[
t(x_i,y_i) = ((\tilde{x}^{i})^T \tilde{y}_i, \tilde{y}_i^2, (\tilde{x}^{i})^T \tilde{x}_i)
\]

with \(\epsilon\)-DP privacy guarantees. This was accomplished
by clamping the data and adding Laplace random error. More precisely, we define
the clamp function \([z] := \min\{\max\{z,-10\}, 10\}\) which truncates a value
\(z\) so that it falls into the interval \([-10,10]\). Furthermore, we let \(\tilde{z} := [z]/10\)
denote the normalized clamped value of \(z\). Thus the clamped statistic is

\[
t(x_i,y_i) = ((\tilde{x}^{i})^T \tilde{y}_i, \tilde{y}_i^2, (\tilde{x}^{i})^T \tilde{x}_i)
\]

and the \(\epsilon\)-DP private statistic is obtained by adding Laplace error
to the unique elements of \ldots{}

data generating process:
\[
\begin{split}
X &\sim N_{2}(\mu, I_2)\\
\mu &= \binom{0.9}{-1.17}
\end{split}
\]

\begin{enumerate}
\def\labelenumi{\arabic{enumi}.}
\item
  \texttt{lik\_f}: Conditional on the table total, the table counts follow a multinomial
  distribution. We can easily draw from this distribution using the
  \texttt{rmultinom} function in the \texttt{base} stats package. Note, in this example,
  the return value of one sample from \texttt{rmultinom} is a \(4 \times 1\) matrix. In order to
  conform with \texttt{dapper\_sample} we must convert the matrix to a vector.

\begin{verbatim}
latent_f <- function(theta) {
  xmat <- MASS::mvrnorm(100 , mu = c(.9,-1.17), Sigma = diag(2))
  y <- cbind(1,xmat) %*% theta + rnorm(1, sd = sqrt(2))
  cbind(y,xmat)
}
\end{verbatim}
\item
  \texttt{post\_f}: Given confidential data \(X\) we can derive the posterior analytically
  using a Dirichlet prior. In this example, we use a flat prior which
  corresponds to Dirch(1) distribution. A sample from the Dirichlet distribution
  can be generated using the gamma distribution via the following relation (INSERT)

\begin{verbatim}
post_f <- function(dmat, theta) {
  x <- cbind(1,dmat[,-1])
  y <- dmat[,1]

  ps_s2 <- solve((1/2) * t(x) %*% x + (1/4) * diag(3))
  ps_m <- ps_s2 %*% (t(x) %*% y) * (1/2)

  MASS::mvrnorm(1, mu = ps_m, Sigma = ps_s2)
}
\end{verbatim}
\item
  \texttt{st\_f}: The complete data can be represented in two ways. Micro vs cell totals.
  (what section to introduce?) This function must return a vector.

\begin{verbatim}
clamp_data <- function(dmat) {
  pmin(pmax(dmat,-10),10) / 10
}

st_f <- function(dmat) {
  sdp_mat <- clamp_data(dmat)
  ydp <- sdp_mat[,1, drop = FALSE]
  xdp <- cbind(1,sdp_mat[,-1, drop = FALSE])

  s1 <- t(xdp) %*% ydp
  s2 <- t(ydp) %*% ydp
  s3 <- t(xdp) %*% xdp

  ur_s1 <- c(s1)
  ur_s2 <- c(s2)
  ur_s3 <- s3[upper.tri(s3,diag = TRUE)][-1]
  c(ur_s1,ur_s2,ur_s3)
}
\end{verbatim}
\item
  \texttt{priv\_f}: Privacy Mechanism
  Guassian white noise is added to each cell total. Hence given
  confidential data \((n_{11}, n_{22}, n_{12}, n_{21})\)
  \[
  \eta(s_{dp} \mid x) = \prod \phi(s_{sd}; n_{ij}, 100^2)
  \]

\begin{verbatim}
#deltaa <- 13
#epsilon <- 10
priv_f <- function(sdp, zt) {
  sum(VGAM::dlaplace(sdp - zt, 0, 13/10, log = TRUE))
}
\end{verbatim}
\end{enumerate}

RUN

\begin{verbatim}
deltaa <- 13
epsilon <- 10
n <- 100
xmat <- MASS::mvrnorm(n, mu = c(.9,-1.17), Sigma = diag(2))
beta <- c(-1.79, -2.89, -0.66)
y <- cbind(1,xmat) %*% beta + rnorm(n, sd = sqrt(2))
z <- st_f(cbind(y,xmat))
z <- z + VGAM::rlaplace(length(z), location = 0, scale = deltaa/epsilon)

dmod <- new_privacy(post_f   = post_f,
                    latent_f = latent_f,
                    priv_f   = priv_f,
                    st_f     = st_f,
                    npar     = 3,
                    varnames = c("beta0", "beta1", "beta2"))

dp_out <- dapper_sample(dmod,
                        sdp = z,
                        niter = 3000,
                        warmup = 1000,
                        chains = 1,
                        init_par = rep(0,3))
\end{verbatim}

\begin{verbatim}
#> # A tibble: 4 x 10
#>   variable  mean median     sd    mad     q5   q95  rhat ess_bulk ess_tail
#>   <chr>    <num>  <num>  <num>  <num>  <num> <num> <num>    <num>    <num>
#> 1 pi_11    0.248  0.249 0.0254 0.0257 0.206  0.288  1.00    191.      388.
#> 2 pi_21    0.334  0.333 0.0261 0.0254 0.291  0.377  1.00    210.      329.
#> 3 pi_12    0.103  0.103 0.0272 0.0271 0.0586 0.149  1.02     76.0     159.
#> 4 pi_22    0.315  0.315 0.0256 0.0261 0.275  0.358  1.00    164.      460.
\end{verbatim}

\includegraphics{dppaper_files/figure-latex/unnamed-chunk-21-1.pdf}
\# Summary

Currently, there is a dearth of software tools privacy researchers can use
to evaluate the impact of privacy mechanisms on statistical analyses.
While there have been tremendous gains in the theoretical aspects of privacy,
the lack of software resources to deploy and work with new privacy techniques has
hampered their adoption. This gap in capability has been noted by several
large industry entities who have begun building software ecosystems for
working with differential privacy. SmartNoise by Microsoft {[}insert citation{]}, for example,
is a tool for generating synthetic data that has differentially private guarantees.
However, the majority of these software tools only address privacy and not the
ensuing analysis, or if it does address the analysis, only for specific models.
Privacy researchers currently lack good tools for evaluating the impact
of privacy mechanisms on a statistical analysis.

Thus DAPPER helps fill an urgent need by providing researchers a way to evaluate
how a particular privacy mechanism might effect a statistical analysis. A notable
feature is its flexibility which allows the users to specify a custom
privacy mechanism. The benefit being that DAPPER can evaluate already
established privacy mechanisms and those that have yet to be discovered.

While DAPPER can be a nice addition to a privacy researcher's tool kit, there
is still considerably more work that can be done. From a methodological standpoint,
DAPPER suffers from slow convergence when the privacy budget is small which is problematic
because this is the scenario where a privacy mechanism can have the greatest impact
on a statistical analysis. On the computational end, without a statistic that satisfies
record additivity, computation time can be unpalatable.

\hypertarget{references}{%
\section*{References}\label{references}}
\addcontentsline{toc}{section}{References}

\hypertarget{refs}{}
\begin{CSLReferences}{1}{0}
\leavevmode\vadjust pre{\hypertarget{ref-TopDown}{}}%
Bureau, U. S. Census. 2023. {``Disclosure Avoidance and the 2020 Census: How the TopDown Algorithm Works.''} 2023. \url{https://www.census.gov/library/publications/2023/decennial/c2020br-04.html}.

\leavevmode\vadjust pre{\hypertarget{ref-Carroll2006}{}}%
Carroll, Raymond J., David Ruppert, Leonard A. Stefanski, and Ciprian M. Crainiceanu. 2006. \emph{Measurement Error in Nonlinear Models}. Chapman; Hall/CRC. \url{https://doi.org/10.1201/9781420010138}.

\leavevmode\vadjust pre{\hypertarget{ref-Dwork2013}{}}%
Dwork, Cynthia, and Aaron Roth. 2013. {``The Algorithmic Foundations of Differential Privacy.''} \emph{Foundations and Trends{\textregistered} in Theoretical Computer Science} 9 (3-4): 211--407. \url{https://doi.org/10.1561/0400000042}.

\leavevmode\vadjust pre{\hypertarget{ref-Gong2022}{}}%
Gong, Ruobin. 2022. {``Transparent Privacy Is Principled Privacy.''} \emph{Harvard Data Science Review}, no. Special Issue 2 (June). \url{https://doi.org/10.1162/99608f92.b5d3faaa}.

\leavevmode\vadjust pre{\hypertarget{ref-Ji2014}{}}%
Ji, Zhanglong, Zachary C. Lipton, and Charles Elkan. 2014. {``Differential Privacy and Machine Learning: A Survey and Review.''} \url{https://arxiv.org/abs/1412.7584}.

\leavevmode\vadjust pre{\hypertarget{ref-Ju2022}{}}%
Ju, Nianqiao, Jordan Awan, Ruobin Gong, and Vinayak Rao. 2022. {``Data Augmentation {MCMC} for Bayesian Inference from Privatized Data.''} In \emph{Advances in Neural Information Processing Systems}, edited by Alice H. Oh, Alekh Agarwal, Danielle Belgrave, and Kyunghyun Cho. \url{https://openreview.net/forum?id=tTWCQrgjuM}.

\leavevmode\vadjust pre{\hypertarget{ref-Karwa2015}{}}%
Karwa, Vishesh, Dan Kifer, and Aleksandra B. Slavković. 2015. {``Private Posterior Distributions from Variational Approximations.''} \url{https://arxiv.org/abs/1511.07896}.

\leavevmode\vadjust pre{\hypertarget{ref-Robert2004}{}}%
Robert, Christian P., and George Casella. 2004. \emph{Monte Carlo Statistical Methods}. \emph{Springer Texts in Statistics}. Springer New York. \url{https://doi.org/10.1007/978-1-4757-4145-2}.

\leavevmode\vadjust pre{\hypertarget{ref-Smith2011}{}}%
Smith, Adam. 2011. {``Privacy-Preserving Statistical Estimation with Optimal Convergence Rates.''} In \emph{Proceedings of the Forty-Third Annual ACM Symposium on Theory of Computing}. STOC'11. ACM. \url{https://doi.org/10.1145/1993636.1993743}.

\leavevmode\vadjust pre{\hypertarget{ref-Yi2017}{}}%
Yi, Grace Y. 2017. \emph{Statistical Analysis with Measurement Error or Misclassification}. \emph{Springer Series in Statistics}. Springer New York. \url{https://doi.org/10.1007/978-1-4939-6640-0}.

\end{CSLReferences}

\bibliography{RJreferences.bib}

\address{%
Kevin Eng\\
Rutgers University\\%
Department of Statistics\\ Piscataway, NJ 08854\\
%
\url{https://www.britannica.com/animal/quokka}\\%
%
\href{mailto:ke157@stat.rutgers.edu}{\nolinkurl{ke157@stat.rutgers.edu}}%
}

\address{%
Jordan A. Awan\\
Purdue University\\%
Department of Statistics\\ West Lafayette, IN 47907\\
%
\url{https://www.britannica.com/animal/quokka}\\%
%
\href{mailto:jawan@purdue.edu}{\nolinkurl{jawan@purdue.edu}}%
}

\address{%
Ruobin Gong\\
Rutgers University\\%
Department of Statistics\\ Piscataway, NJ 08854\\
%
\url{https://www.britannica.com/animal/quokka}\\%
%
\href{mailto:ruobin.gong@rutgers.edu}{\nolinkurl{ruobin.gong@rutgers.edu}}%
}

\address{%
Nianqiao Phyllis Ju\\
Purdue University\\%
Department of Statistics\\ West Lafayette, IN 47907\\
%
\url{https://www.britannica.com/animal/quokka}\\%
%
\href{mailto:nianqiao@purdue.edu}{\nolinkurl{nianqiao@purdue.edu}}%
}

\address{%
Vinayak A. Rao\\
Purdue University\\%
Department of Statistics\\ West Lafayette, IN 47907\\
%
\url{https://www.britannica.com/animal/quokka}\\%
%
\href{mailto:varao@purdue.edu}{\nolinkurl{varao@purdue.edu}}%
}
