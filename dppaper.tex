% !TeX root = RJwrapper.tex
\title{gdpR: An R Package for studying differentially private algorithms}


\author{by Jordan A. Awan, Kevin Eng, Robin Gong, Nianqiao Phyllis Ju, and Vinayak A. Rao}

\maketitle

\abstract{%
This paper serves as a reference and introduction on using the \(gdpR\) R package. The goal of this package is to provide some tools for exploring the impact of different privacy regimes on a Bayesian analysis. A strength of this framework is the ability to target the exact posterior in settings where the likelihood is too complex to analytically express.
}

\hypertarget{introduction}{%
\section{Introduction}\label{introduction}}

The ease and pervasiveness of modern data collection technologies has raised
concerns about data privacy. (Dwork and Roth 2013) introduced the differential privacy
framework as a means to rigorously define privacy. The framework has lead to the
development of many ``privitized'\,' versions of existing statistical methods. The
process of privitizing usually consist of introducing random noise in someway using
a known distribution.

\hypertarget{using}{%
\section{\texorpdfstring{Using \CRANpkg{gdpR}}{Using }}\label{using}}

\hypertarget{sampling}{%
\subsection{Sampling}\label{sampling}}

The main function in \pkg{gplsim} is the \texttt{gdp\_sample} function. The call
signature of the function is:

\begin{verbatim}
gdp_sample(data_model, sdp, nobs, init_par, niter = 2000, warmup = floor(niter / 2),
           chains = 1, varnames = NULL)
\end{verbatim}

The three required inputs into \texttt{gdp\_sample} function are the privacy model (\texttt{data\_model}), the value
of the observed privatized statistic (\texttt{sdp}), and the total number of observations
in the complete data (\texttt{nobs}) {[}MAKE SURE NOTATION IS INTRODUCED{]}. The \CRANpkg{gdpR}
package is best suited for problems where the complete data can be represented in
tabular form. This is because internally, it is represented as a matrix.

The optional arguments are the number of mcmc draws (\texttt{niter}), the
burn in period (\texttt{warmup}), number of chains (\texttt{chains}) and character
vector that names the parameters. Running multiple chains can be done in parallel
using the \CRANpkg{furrr} package. Additionally, progress can be monitored
using the \CRANpkg{progressr} package.

The \texttt{data\_model} input is a \texttt{privacy}
object that can be constructed using the \texttt{new\_privacy} constructor. The
process of constructing a \texttt{privacy} object will be discussed in the next section.

\hypertarget{privacy-model}{%
\subsection{Privacy Model}\label{privacy-model}}

Creating a privacy model is done using the \texttt{new\_privacy} constructor. The
main arguments consist of the four components as outlined in the methodology
section.

\begin{verbatim}
new_privacy(post_smpl = NULL, lik_smpl = NULL, ll_priv_mech = NULL,
            st_calc = NULL, add = FALSE, npar = NULL)
\end{verbatim}

The internal implementation of the DA algorithm in \texttt{gdp\_sample} requires
some care in how each component in constructed.

\begin{itemize}
\item
  \texttt{lik\_smpl} is an R function that samples from the likelihood. Its
  call signature should be \texttt{lik\_smpl(theta)} where \texttt{theta} is a vector
  representing the likelihood model parameters being estimated. This function
  must work with the supplied initial parameter provide in the \texttt{init\_par}
  argument of \texttt{gdp\_sample}. The sampler need not be vectorized and vectorizing
  the sampler will not add any speed benefits.
\item
  \texttt{post\_smpl} is a function which represents the posterior sampler. It should
  have the call signature \texttt{post\_smpl(dmat,\ theta)}. Where \texttt{dmat} is the
  complete data. This sampler can be generated by wrapping mcmc samplers generated from other R packages
  (e.g.~\CRANpkg{rstan}, \CRANpkg{fmcmc}, \CRANpkg{adaptMCMC}).
  If using this approach, it is recommended to avoid using packages such as \CRANpkg{mcmc}
  whose implementation clashes with \texttt{gdp\_sample}. In the case of \CRANpkg{mcmc},
  the Metropolis-Hastings loop is implemented in C which incurs a very large overhead
  in \texttt{gdp\_sample} since it is reinitialized every iteration. In general, repeatedly calling
  an R function that hooks into C code is slow. (NOT QUITE ACCURATE FIX LATER)
\end{itemize}

\hypertarget{example}{%
\section{Example}\label{example}}

\hypertarget{background}{%
\section{Background}\label{background}}

Some packages on interactive graphics include \CRANpkg{plotly} (Sievert 2020) that interfaces with Javascript for web-based interactive graphics, \CRANpkg{crosstalk} (Cheng and Sievert 2021) that specializes cross-linking elements across individual graphics. The recent R Journal paper \CRANpkg{tsibbletalk} (Wang and Cook 2021) provides a good example of including interactive graphics into an article for the journal. It has both a set of linked plots, and also an animated gif example, illustrating linking between time series plots and feature summaries.

\hypertarget{customizing-tooltip-design-with}{%
\section{\texorpdfstring{Customizing tooltip design with \pkg{ToOoOlTiPs}}{Customizing tooltip design with }}\label{customizing-tooltip-design-with}}

\hypertarget{summary}{%
\section{Summary}\label{summary}}

We have displayed various tooltips that are available in the package \pkg{ToOoOlTiPs}.

\hypertarget{references}{%
\section*{References}\label{references}}
\addcontentsline{toc}{section}{References}

\hypertarget{refs}{}
\begin{CSLReferences}{1}{0}
\leavevmode\vadjust pre{\hypertarget{ref-crosstalk}{}}%
Cheng, Joe, and Carson Sievert. 2021. \emph{{crosstalk}: Inter-Widget Interactivity for HTML Widgets}. \url{https://CRAN.R-project.org/package=crosstalk}.

\leavevmode\vadjust pre{\hypertarget{ref-Dwork2013}{}}%
Dwork, Cynthia, and Aaron Roth. 2013. {``The Algorithmic Foundations of Differential Privacy.''} \emph{Foundations and Trends{\textregistered} in Theoretical Computer Science} 9 (3-4): 211--407. \url{https://doi.org/10.1561/0400000042}.

\leavevmode\vadjust pre{\hypertarget{ref-plotly}{}}%
Sievert, Carson. 2020. \emph{{Interactive Web-Based Data Visualizatio}n with r, Plotly, and Shiny}. Chapman; Hall/CRC. \url{https://plotly-r.com}.

\leavevmode\vadjust pre{\hypertarget{ref-RJ-2021-050}{}}%
Wang, Earo, and Dianne Cook. 2021. {``Conversations in Time: Interactive Visualisation to Explore Structured Temporal Data.''} \emph{The R Journal}. \url{https://doi.org/10.32614/RJ-2021-050}.

\end{CSLReferences}

\bibliography{RJreferences.bib}

\address{%
Jordan A. Awan\\
Purdue University\\%
Department of Statistics\\ West Lafayette, IN 47907\\
%
\url{https://www.britannica.com/animal/quokka}\\%
%
\href{mailto:jawan@purdue.edu}{\nolinkurl{jawan@purdue.edu}}%
}

\address{%
Kevin Eng\\
Rutgers University\\%
Department of Statistics\\ Piscataway, NJ 08854\\
%
\url{https://www.britannica.com/animal/quokka}\\%
%
\href{mailto:ke157@stat.rutgers.edu}{\nolinkurl{ke157@stat.rutgers.edu}}%
}

\address{%
Robin Gong\\
Rutgers University\\%
Department of Statistics\\ Piscataway, NJ 08854\\
%
\url{https://www.britannica.com/animal/quokka}\\%
%
\href{mailto:ruobin.gong@rutgers.edu}{\nolinkurl{ruobin.gong@rutgers.edu}}%
}

\address{%
Nianqiao Phyllis Ju\\
Purdue University\\%
Department of Statistics\\ West Lafayette, IN 47907\\
%
\url{https://www.britannica.com/animal/quokka}\\%
%
\href{mailto:nianqiao@purdue.edu}{\nolinkurl{nianqiao@purdue.edu}}%
}

\address{%
Vinayak A. Rao\\
Purdue University\\%
Department of Statistics\\ West Lafayette, IN 47907\\
%
\url{https://www.britannica.com/animal/quokka}\\%
%
\href{mailto:varao@purdue.edu}{\nolinkurl{varao@purdue.edu}}%
}
