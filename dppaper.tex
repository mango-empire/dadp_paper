% !TeX root = RJwrapper.tex
\title{gdpR: An R Package for studying differentially private algorithms}


\author{by Jordan A. Awan, Kevin Eng, Robin Gong, Nianqiao Phyllis Ju, and Vinayak A. Rao}

\maketitle

\abstract{%
An abstract of less than 150 words.
}

\hypertarget{introduction}{%
\section{Introduction}\label{introduction}}

Interactive data graphics provides plots that allow users to interact them. One of the most basic types of interaction is through tooltips, where users are provided additional information about elements in the plot by moving the cursor over the plot.

This paper will first review some R packages on interactive graphics and their tooltip implementations. A new package \CRANpkg{ToOoOlTiPs} that provides customized tooltips for plot, is introduced. Some example plots will then be given to showcase how these tooltips help users to better read the graphics.

\hypertarget{using}{%
\section{\texorpdfstring{Using \CRANpkg{gdpR}}{Using }}\label{using}}

\hypertarget{sampling-function}{%
\subsection{Sampling function}\label{sampling-function}}

The main function in the \CRANpkg{gdpR} is the \texttt{gdp\_sample} function. The call
signature of the function is:

\texttt{gdp\_sample(data\_model,\ sdp,\ nobs,\ init\_par,\ niter\ =\ 2000,\ warmup\ =\ floor(niter\ /\ 2),\ \ \ \ \ \ \ \ \ \ \ \ chains\ =\ 1,\ varnames\ =\ NULL)}

The three required inputs into \texttt{gdp\_sample} function are the privacy model (\texttt{data\_model}), the value
of the observed privatized statistic (\texttt{sdp}), and the total number of observations
in the complete data (\texttt{nobs}) {[}MAKE SURE NOTATION IS INTRODUCED{]}. The \CRANpkg{gdpR}
package is best suited for problems where the complete data can be represented in
tabular form. This is because internally, it is represented as a matrix.

The optional arguments are the number of mcmc draws (\texttt{niter}), the
burn in period (\texttt{warmup}), number of chains (\texttt{chains}) and character
vector that names the parameters. Running multiple chains can be done in parallel
using the \CRANpkg{furrr} package. Additionally, progress can be monitored
using the \CRANpkg{progressr} package.

\hypertarget{creating-a-privacy-object}{%
\subsection{\texorpdfstring{Creating a \texttt{privacy} object}{Creating a privacy object}}\label{creating-a-privacy-object}}

Creating a \texttt{privacy} object is done via the \texttt{new\_privacy} function.

\texttt{new\_privacy(post\_smpl\ =\ NULL,\ lik\_smpl\ =\ NULL,\ ll\_priv\_mech\ =\ NULL,\ \ \ \ \ \ \ \ \ \ \ \ \ st\_calc\ =\ NULL,\ add\ =\ FALSE,\ npar\ =\ NULL)}

. The \texttt{data\_model} input is a \texttt{privacy}
object that can be constructed using the \texttt{new\_privacy} constructor. The
process of constructing a \texttt{privacy} object will be discussed in the next section.

Parallel computing is implemented
using the \CRANpkg{futures} package.

The \texttt{new\_privacy} function creates a new data model for input in the
\texttt{gdp\_sample} function.

\hypertarget{background}{%
\section{Background}\label{background}}

Some packages on interactive graphics include \CRANpkg{plotly} (Sievert 2020) that interfaces with Javascript for web-based interactive graphics, \CRANpkg{crosstalk} (Cheng and Sievert 2021) that specializes cross-linking elements across individual graphics. The recent R Journal paper \CRANpkg{tsibbletalk} (Wang and Cook 2021) provides a good example of including interactive graphics into an article for the journal. It has both a set of linked plots, and also an animated gif example, illustrating linking between time series plots and feature summaries.

\hypertarget{customizing-tooltip-design-with}{%
\section{\texorpdfstring{Customizing tooltip design with \pkg{ToOoOlTiPs}}{Customizing tooltip design with }}\label{customizing-tooltip-design-with}}

\pkg{ToOoOlTiPs} is a packages for customizing tooltips in interactive graphics, it features these possibilities.

\hypertarget{summary}{%
\section{Summary}\label{summary}}

We have displayed various tooltips that are available in the package \pkg{ToOoOlTiPs}.

\hypertarget{references}{%
\section*{References}\label{references}}
\addcontentsline{toc}{section}{References}

\hypertarget{refs}{}
\begin{CSLReferences}{1}{0}
\leavevmode\vadjust pre{\hypertarget{ref-crosstalk}{}}%
Cheng, Joe, and Carson Sievert. 2021. \emph{{crosstalk}: Inter-Widget Interactivity for HTML Widgets}. \url{https://CRAN.R-project.org/package=crosstalk}.

\leavevmode\vadjust pre{\hypertarget{ref-plotly}{}}%
Sievert, Carson. 2020. \emph{{Interactive Web-Based Data Visualizatio}n with r, Plotly, and Shiny}. Chapman; Hall/CRC. \url{https://plotly-r.com}.

\leavevmode\vadjust pre{\hypertarget{ref-RJ-2021-050}{}}%
Wang, Earo, and Dianne Cook. 2021. {``Conversations in Time: Interactive Visualisation to Explore Structured Temporal Data.''} \emph{The R Journal}. \url{https://doi.org/10.32614/RJ-2021-050}.

\end{CSLReferences}

\bibliography{RJreferences.bib}

\address{%
Jordan A. Awan\\
Purdue University\\%
Department of Statistics\\ West Lafayette, IN 47907\\
%
\url{https://www.britannica.com/animal/quokka}\\%
%
\href{mailto:jawan@purdue.edu}{\nolinkurl{jawan@purdue.edu}}%
}

\address{%
Kevin Eng\\
Rutgers University\\%
Department of Statistics\\ Piscataway, NJ 08854\\
%
\url{https://www.britannica.com/animal/quokka}\\%
%
\href{mailto:ke157@stat.rutgers.edu}{\nolinkurl{ke157@stat.rutgers.edu}}%
}

\address{%
Robin Gong\\
Rutgers University\\%
Department of Statistics\\ Piscataway, NJ 08854\\
%
\url{https://www.britannica.com/animal/quokka}\\%
%
\href{mailto:ruobin.gong@rutgers.edu}{\nolinkurl{ruobin.gong@rutgers.edu}}%
}

\address{%
Nianqiao Phyllis Ju\\
Purdue University\\%
Department of Statistics\\ West Lafayette, IN 47907\\
%
\url{https://www.britannica.com/animal/quokka}\\%
%
\href{mailto:nianqiao@purdue.edu}{\nolinkurl{nianqiao@purdue.edu}}%
}

\address{%
Vinayak A. Rao\\
Purdue University\\%
Department of Statistics\\ West Lafayette, IN 47907\\
%
\url{https://www.britannica.com/animal/quokka}\\%
%
\href{mailto:varao@purdue.edu}{\nolinkurl{varao@purdue.edu}}%
}
